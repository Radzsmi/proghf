Fimtár

A feladatom\+:

A feladatomnak a \char`\"{}\+Filmtár\char`\"{} nevű programot válaszotta a feladatötleketből,egy nyivántartó renszert fogok létrehozni amiben filmeket tárolok.\+Minden film rendelkezik majd címmel,lejátszási idővel és kiadási évvel.\+Továbbá filmeket meg lehet majd különböztetni kategória alapján.(pl.\+családi,dokumentum)Ezek bónusz adatokkal is rendelkeznek.

Funkciók\+: -\/Listázás -\/Hozzáadás -\/Keresés -\/Törlés

A nyilvántartásra használt program működését egy konzolos feöleten lehet majd irányítani,számokat megadva az egyes funkciók eléréséhez,más adatot nem fogad el a program.

A programot egy menüben lehet majd irányítani ahova minden művelet után visszadob a rendszer,onnan lehet majd kiválasztani a következő lépést.

A tesztelést listázásra,keresésre,törlésre fogom kipróbálni.

Pontosított feladatspecifikáció

A feladat egy nyilvántartó program megírása amely megnyitáskor a forrás szöveges fájlból olvassa ki az adatokat,és dinamikusan tárolja őket objektumokba rendezve a program futásának végéig amikor visszaírja őket a szöveges fájlba. Minden műveletet (módosítás,törlés,hozzáadás) ezen a dinamikusan foglalt objektumon hajt végre a program. Az adatok megadásánál a megfelelő adattípúsoknak megfelelően kell adatokat megadni,vagyis például a megjelenés évének csak számokat fogad el a program. Amennyiben nem létezik/üres a txt file,abban az esetben a program létrehozza a futásának végén.

Objektum terv

Az adatok tárolását a \mbox{\hyperlink{class_filmek}{Filmek}} osztály egyik objektumában fogom megvalósítani.\+Minden filmet vagy sima,családi vagy dokumentumfilmként fog tárolni a program.\+A \mbox{\hyperlink{class_filmek}{Filmek}} osztály tároljak az elemek számát így könnyebb kezelni a dinamikus foglalásokat és az adatok hozzáadását. A program elkészítése során igyekszem minnél könnyebben bővíthető objektumszerkezetet létrehozni,ennek érdekében egy új objektum megírásához mindössze konstruktor és getter,setter lesz szükséges,és 1 sor változtatás a kódban.

Algoritmusok

A feladatom egyik legösszetettebb algoritmusa az adatok beolvasása amit egy külön föggvény fog megvalósítani.\+A függvény argumentuma egy filmek objektum amit a mainben hoz létre a program. A beolvasás a txtfájl mintája szerint fog történni és annak egyedi mintája szerint minden sorból kiolvassa az odaillő elemeket.\+Minden beolvasás elején a txtfájlban van egy betű ami a film típúsára utal,ennek megfelelően megtudja a program határozni az egyes filmekhez tartozó adatok számát. A filmek bekérése az eof-\/ig történik.\+Addig minden filmből objektumot hoz létre a program amit behelyez a mainből kapott filmek típusú objektumba. A keresés filmek címei között zajlik,amelyekből azokat tekinti találatnak melyekben szerepel ugyanaz a részlet,kis és nagybetűket egyként kezeli.

A tesztprogram működése

A tesztprogram egy előre elkészített eseménysorozatot fog lejátszani és ezek helyességét fogja kiértékelni.\+A tesztesetek a program működését fogják szimulálni,mivel az adatok beolvasása nem kivételt hanem újbóli bekérést eredményez ezért azokra nem terjed ki.(viszont ezeket is megvalósítja a program) A tesztet a jporta definiálásával lehet elindítani de lehetőség lesz a program kipróbálásra is.

A tesztprogram lefuttat a lehető legtöbb esetet,de g\+\_\+test segítségével csak az előre megírt tesztesetek futnak.\+Viszont a nem ellenőrzött tesztesetek eredménye az hogy a program majdnem minden függvénye meghívódik és a filmlista.\+txt változatlan marad hiszen a teszt a törlést is meghívja.(Erre a hibás bekérések miatti ellenőrzésre van szükség.)

A dokumentáció

A doxygen program segítségével.\+Ékezetek nélkül,mivel különböző karaktereket nem jelenít meg a program.

Bővíthetőség

A program bővítéséhez a \mbox{\hyperlink{filmtipusok_8h_source}{filmtipusok.\+h}}-\/ban kell egy új classt definiálni getter,setter,konstruktorral és a segedfgvek.\+cpp-\/ben a custromconst fgvben beleírni hogy a megfelelő jelnél a jó konstruktor hívódjon meg. 